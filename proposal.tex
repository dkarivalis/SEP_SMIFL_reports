\documentclass[11pt,letterpaper,oneside]{memoir}
\usepackage{smiflstyle}

% define custom title layout (slightly temperamental)
\title{%
{\color{color2} \hrule}\vspace{1cm}
\Huge{\color{color1} Project Proposal:\\The Paramount Investments League %temporary name, we should change this
\vspace{1cm}
{\color{color2} \hrule}\vspace{1cm}}
\Large{ \color{color2} Software Engineering\\
14:332:452}
}

% define custom author layout (highly temperamental)
\author{\huge{\color{color0}Team 1:\\}\vskip.1in
\Large{\href{mailto:david.patrzeba@gmail.com}{David Patrzeba}\\
\href{mailto:eric.jacob.10@gmail.com}{Eric Jacob}\\
\href{mailto:evanarbeitman@gmail.edu}{Evan Arbeitman}\\
\href{mailto:christopher.a.mancuso@gmail.com}{Christopher Mancuso}\\
\href{mailto:dkarivalis@gmail.edu}{David Karivalis}\\
\href{mailto:jdlziegler@gmail.com}{Jesse Ziegler}}}

\date{\today}

\usepackage{hyperref}
\begin{document}
%\maketitle % don't use this, will break the Author section
\titleGM    % use this instead, defined to avoid problem

Hyperlinks:\\
\begin{center}
\href{http://192.241.248.91}{Webapp Link}\\
\href{https://github.com/dkarivalis/SEP_SMIFL}{Project Repository}\\
\href{https://github.com/dkarivalis/SEP_SMIFL_reports}{Reports Repository}\\
\end{center}

Revision History:
\begin{longtable}{|p{1.6in}|p{2.6in}|}
\hline
{\large \color{color1}Version No.}&{\large \color{color1}Date of Revision} \\ \hline
v.1&1/26/2014  \\ \hline 
\end{longtable}

\pagebreak  % flush the next page
\tableofcontents % create TOC
\setlength\parindent{0pt}
%%%%%%%%%%%%%%%%%%%%%%%%%%%%%%%%%%%%%%%%%%%%%%%%%%%%%%%%%%%%%%%%%%%%%%%%%%%%%%%%%%%%%%%%%%%%%%
\chapter{Team Profile}

Team 1 will be working on \href{http://ece.rutgers.edu/~marsic/books/SE/projects/}
{Project 5: Stock Market Investment Fantasy League} and they have named their project
``The Paramount Investments League''. This project is intended to serve
as an exercise in software engineering focusing on building experience in the design,
architecture, construction, test, and maintenance of a small-to-mid sized complex
software application.\\

At this time a project lead has not been elected, but David Patrzeba will be acting
as technical lead.  All members will have input on the decisions of the team and
communication is being facilitated by a project mailing list, git repositories,
a wiki, and google+ hangouts.\\

\section{David Patrzeba}

David is proficient with the Java, C, and C++ languages, RESTful APIs, SQL, and is
highly familiar with iterative software design and object oriented design patterns.
David also has experience with Android development, relational database schema, and
user experience.  David will be acting as a technical lead on the project.

\section{Eric Jacob}

Eric is proficient with the Java, C and C++ languages, SQL, and Python. Eric
also has experience with Bash scripting, JDBC, and the JavaMail API. Eric also
has some familiarity with Java Servlet API and Java Persistence API.


\section{Evan Arbeitman}

Evan is proficient with the C++ language. Evan is also familiar with some PHP, Python,
Bash Shell Scripting and Python languages.

\section{Jesse Ziegler}

Jesse is proficient in C and C++. Jesse is familiar with the concepts of object
oriented programming and some experience in HTML/Javascript. He is also familiar
with concepts of stock exchange investing and virtual trading systems.

\section{David Karivalis}

David is proficient with the PHP, Java language. David is familiar with C, JavaScript, and HTML/CSS.
David has experience in iOS and Android development, user experience, and photoshop.  David will be
the UI lead.

\section{Christopher Mancuso}

Chris is proficient in the C++ language with familiarity in HTML/CSS, PHP, Bash, and JavaScript. He
also has knowledge on remote servers, vmware, vsphere, wamp, and other networking applications.

%%%%%%%%%%%%%%%%%%%%%%%%%%%%%%%%%%%%%%%%%%%%%%%%%%%%%%%%%%%%%%%%%%%%%%
\chapter{Project Proposal}
\label{proposal}
Team 1 has elected to work on \href{http://ece.rutgers.edu/~marsic/books/SE/projects/}
{Project 5: Stock Market Investment Fantasy League} with the goal of implementing a
web application to service a core audience of novice investors in introducing them to
the ins and outs of tradeable assets.  Novice investors come from all ages and backgrounds
but tend to fall into the 16-30 year old crowd.  By "game-ifying" the action of trading
stocks; by allowing users to collect achievments and be rewarded in interesting ways; we
plan to maintain their interest in the game and grow the user base while simultaneously
teaching them basic investment strategies.

\section{Resgistration}

The end user should be able to register and login to the system in a simple and straight forward
manner requesting the least amount of information necessary in order to start using the system.

\section{Social Media Integration}

End users should be able to push messages to their social media personality that indicate
recent trades that they have made or achievments that they have earned.

\section{Transactions Ticker}

All site vistors should see a ticker of the most recent trades scroll across the bottom of their
screen at a speed such that they can read and process the information easily much like seen on
CNBC.

\section{Unified Interfaces}

The end should experience a unified experience across mobile, tablet, and desktop browsers.
The customer should be able to use the major modern browsers Firefox, Chrome, Safari, and
Internet Explorer.

\section{Portfolio Management}

The end user should be able to place orders to buy, sell, short, stop, and limit, on any
tradeable asset available on the NYSE and the NASDAQ. The user should be able to
cancel any pending orders that have not gone through (eg, a limit order that hasn't triggered).

\section{Graphs and News}

End users should be presented with a news feed related to his/her portfolio.
Additionally they should be able to manipulate graphs in order to compare
performance of a variety of tradeable assets.

\section{Email Updates}

The end user should be able to receive email updates at a frequency and granularity that they choose.

\section{Educational Interfaces}

End users should be able to mouse over investment terms (eg, P/E ratio) and see a pop up
dialog with a brief description and links to internal and external resources.

\section{Leagues}

End users should be able to create, modify, and participate in leagues that agree to a customizable
set of rules to determine a winner (eg, the first to double their money).  Leagues should be able
to be set up by any end user and be made public or private.

\section{Achievments}

End users should receive recognition of achievments accomplished (eg, earn 10\% in a month) and
be rewarded with additional play money, stocks, and other novel rewards.

%%%%%%%%%%%%%%%%%%%%%%%%%%%%%%%%%%%%%%%%%%%%%%%%%%%%%%%%%%%%%%%%%%%%%%%%%%%%%%%%%%%%%%%%%%%%%%%%%%%%
\chapter{Product Ownership}
The project has identified 10 core pieces of functionality which will be implemented by a team of two
with the first name listed set as the go to man for that given piece of functionality.  Every team
member will be assigned a minimum of 3 pieces of functionality to be responsible for and will be
lead on no more than two pieces of functionality.

\section{Registration Functionality}

Evan Arbeitman and Christopher Mancuso

\section{Social Media Integration Functionality}

Jesse Ziegler and Eric Jacob

\section{Transaction Ticker Functionality}

David Karivalis and Evan Arbeitman

\section{Unified Interface Functionality}

David Karivalis and Eric Jacob

\section{Portfolio Managment Functionality}

David Patrzeba and David Karivalis

\section{Graphs and News Functionality}

Christopher Mancuso and Jesse Zeigler

\section{Email Update Functionality}

Evan Arbeitman and Jesse Zeigler

\section{Educational Interface Functionality}

Eric Jacob and Christopher Mancuso

\section{League Functionality}

Jesse Zeigler and David Patrzeba

\section{Acheivments Function}

David Patrzeba and David Karivalis

\end{document}

