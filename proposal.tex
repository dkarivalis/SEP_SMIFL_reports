\documentclass[11pt,letterpaper,oneside]{memoir}
\usepackage{smiflstyle}

% define custom title layout (slightly temperamental)
\title{%
{\color{color2} \hrule}\vspace{1cm}
\Huge{\color{color1} Project Proposal:\\Bulls and Bears %temporary name, we should change this
\vspace{1cm}
{\color{color2} \hrule}\vspace{1cm}}
\Large{ \color{color2} Software Engineering\\
14:332:452}
}

% define custom author layout (highly temperamental)
\author{\huge{\color{color0}Team 1:\\}\vskip.1in
\Large{\href{mailto:david.patrzeba@gmail.com}{David Patrzeba}\\
\href{mailto:eric.jacob.10@gmail.com}{Eric Jacob}\\
\href{mailto:evanarbeitman@gmail.edu}{Evan Arbeitman}\\
\href{mailto:christopher.a.mancuso@gmail.com}{Christopher Mancuso}\\
\href{mailto:dkarivalis@gmail.edu}{David Karivalis}\\
\href{mailto:jdlziegler@gmail.com}{Jesse Ziegler}}}

\date{\today}

\usepackage{hyperref}
\begin{document}
%\maketitle % don't use this, will break the Author section
\titleGM    % use this instead, defined to avoid problem

Hyperlinks:\\
\begin{center}
\href{http://192.241.248.91}{Webapp Link}\\
\href{https://github.com/dkarivalis/SEP_SMIFL}{Project Repository}\\
\href{https://github.com/dkarivalis/SEP_SMIFL_reports}{Reports Repository}\\
\end{center}

Revision History:
\begin{longtable}{|p{1.6in}|p{2.6in}|}
\hline
{\large \color{color1}Version No.}&{\large \color{color1}Date of Revision} \\ \hline
v.1&1/26/2014  \\ \hline 
\end{longtable}

\pagebreak  % flush the next page
\tableofcontents % create TOC


%%%%%%%%%%%%%%%%%%%%%%%%%%%%%%%%%%%%%%%%%%%%%%%%%%%%%%%%%%%%%%%%%%%%%%
\chapter{Project Proposal}
\label{proposal}
Team 1 has elected to work on \href{http://ece.rutgers.edu/~marsic/books/SE/projects/}
{Project 5: Stock Market Investment Fantasy League} with the goal of implementing a
web application. We intend to use a RESTful API for executing all requests by the end user
allowing easy expansion to the desktop and mobile application domains. We are deploying to
a \href{http://www.digitialocean.com}{DigitalOcean} Virtual Private Server (droplet) which
will allow us to scale as necessary, both vertically and horizontally.\\

The goal of our web application is to act as the initial and primary interface to our backend
services. These will include the ability to conduct buy, sell, short, stop, and limit orders on
at least the NYSE and NASDAQ stock exchanges.  We also plan to support global leader boards,
individual leagues with goals (eg. first to double their money, first to gain 8\% in a day, etc...),
and global achievements (similar to xBox achievments).\\

Because our project will be built using RESTful principles, it will be able to act as a
platform that can be extended by third parties to implement new and innovative features.
Some scenarios include trading tutorials, and stock prediction integration.  These are not
of our primary concern.

\section{Market Models}

The fundamental data modeled by stock market fantasy leagues is the stock market. There
are many and varied options available to traders, the most fundamental of which are
buy, sell, and short. We fully intend to support these trading operations. Supporting 
more advanced trading features, such as trading on margin, is also a goal. 

Group 6 in 2012 implemented an interesting feature: user-defined mutual and hedge funds.
They created a model in which a single user could create a portfolio in which other 
users could invest, and follow the gains and losses of that fund. This exposes an interesting
notion, that of following the trades of other users in a given league. 

We therefore propose functionality of visit-able trader profile pages for 
users within a given league. In this model, should be able to track the trades 
and portfolio performances of their peers, while not being able to execute trades 
on their behalf. This should promote the competition of the league.

\section{Social Media Integration}

In recent years, some groups, such as Group 3 in 2012 and Group 6 in 2011, have provided
Facebook interfaces for their applications. These come in two variants: making Facebook
the \emph{de facto} portal for an application; and providing the option of using Facebook
as an authentication system. We note that Group 6 failed to achieve the Facebook authentication
by their deadline. Therefore, at this time we plan to strike a more conservative path and
create a stand-alone website with as few external dependencies as possible.

On the other hand, incorporating modular external features is a definite possibility. We note
that Group 3 created a Twitter interface for their trading application, through which users
were able to ``tweet'' trades. Such advanced functionality is an option, but should be considered
an additional feature, and as such we can consider adding this module at a later point.

\section{Transaction Feed}

Another feature inspired by Twitter and Facebook is the implementation of a user transaction
news feed. A feature which could be integrated into the ``social'' aspect of the leagues
is the ability to have a feed added to certain relevant pages, containing updates about 
which transactions other players are making. This could offer an interesting study as to how
players react to real-time information about player positions.

\section{Data Source}

After conducting market research, we decided to use Yahoo! Finance's HTTP 
interface for 20-minutes-delayed financial markets data, as have groups in 
previous years. Though other services
exist (e.g. 
\href{http://www.bloomberg.com/enterprise/enterprise_products/data_optimization/data_feeds/}{Bloomberg},
\href{http://www.financialcontent.com}{Financial Content}), they are nearly all paid-subscription models. 
The few services which offer free or ``freemium'' services (e.g. 
\href{http://eoddata.com}{eoddata}) have other unacceptable limitations, such as a 
lack of live data. Despite its age, Yahoo! Finance API is currently still the only 
free, versatile finance API. This is despite its unofficial ceiling of daily requests,
as we do not expect to receive sufficient volume of traffic as to approach this ceiling.

\section{Mobile and Tablet Interfaces}

One area in which we intend to differ from teams of previous years is with a unified
mobile and desktop website experience. Due to recent advances in web site design, 
known as ``responsive design'', it is possible to design a single site which is accessible 
from mobile, tablet, and traditional web browsers. Various CSS libraries provide
these designs styles, including
\href{http://twitter.github.com/bootstrap/}{Twitter Bootstrap}
and
\href{http://getskeleton.com}{Skeleton}.
Additionally, enhancements in mobile 
browser capabilities enable the use of Javascript, which is now supported universally
by all modern mobile browsers. These changes opens up availability of our team to dedicate 
more resources towards core site functionality.

\section{Interactive Portfolio Graphs}

Another area in which we intend to offer improvements over features provided in previous years
is that of interactive portfolio graphs. Highcharts JS provides the 
\href{http://www.highcharts.com/products/highstock}{HighStock} dynamic interactive graphing API.
This particular library is designed with financial modeling in mind, as it includes 
dynamic tooltips, time scrolling and time zooming. Thus we can present a user's entire 
portfolio's performance in a single object, for concise control and trend analytics. Other 
options include graphing propsective investments and graphing multiple investments on a 
single graph for comparison.

\section{Periodic Portfolio Email Updates}

A feature often presented to commercial investors is periodic performance update emails.
Like Group 3 in 2011, we would like to implement an email system which
will periodically inform users of their portfolio performance in various leagues, as well
as any gains or losses they may have incurred since the last update. We note that Group 3
failed to model performance of their portfolios, instead simply regurgitating the day's-end
values. An important improvement will be offering the ability to compare to previous trading
data.

\section{Educational Tutorial}

There are two target demographics for this application: students and novice investors. As a
result, we find it important to the success of our project that we include an interactive
tutorial to educate our users in both the basic and more advanced aspects of investing
and finance. This will greatly broaden our appeal to users and make a marked improvement over
previous interactions by combining social, educational, and entertaining elements to create
a rich and captivating experience.

%%%%%%%%%%%%%%%%%%%%%%%%%%%%%%%%%%%%%%%%%%%%%%%%%%%%%%%%%%%%%%%%%%%%%%%%%%%%%%%%%%%%%%%%%%%%%%
\chapter{Team Profile}

Team 1 will be working on \href{http://ece.rutgers.edu/~marsic/books/SE/projects/}
{Project 5: Stock Market Investment Fantasy League} and have named their project
''Bulls and Bears'' for the tentative future. This project is intended to serve
as an exercise in software engineering focusing on building experience in the design,
architecture, construction, test, and maintenance of a small-to-mid sized complex
software application.\\

At this time a project lead has not been elected, but David Patrzeba will be acting
as technical lead.  All members will have input on the decisions of the team and
communication is being facilitated by a project mailing list, git repositories,
a wiki, and google+ hangouts.\\

\section{David Patrzeba}

David is proficient with the Java, C, and C++ languages, RESTful APIs, SQL, and is
highly familiar with iterative software design and object oriented design patterns.
David also has experience with Android development, relational database schema, and
user experience.  David will be acting as a technical lead on the project.

\section{Eric Jacob}

Eric is proficient with the Java, C, and C++ languages, and SQL.  Eric also has
experience with Bash scripting and Python.

\section{Evan Arbeitman}

Nick is familiar with the C++ and PHP languages, Bash scripting and Python.

\section{Jesse Ziegler}

Jesse has experience with the C and C++ languages.

\section{David Karivalis}

David is proficient with the PHP, Java language. David is familiar with C, JavaScript, and HTML/CSS.
David has experience in iOS and Android development, user experience, and photoshop.  David will be
the UI lead.

\section{Christopher Mancuso}

Chris is familar with the C++ language.

%%%%%%%%%%%%%%%%%%%%%%%%%%%%%%%%%%%%%%%%%%%%%%%%%%%%%%%%%%%%%%%%%%%%%%%%%%%%%%%%%%%%%%%%%%%%%%%%%%%%
\chapter{Product Ownership}

We have decided to divvy up the group in such a way that each function is claimed by a
particular team member. In this way, we allow for easy traceability for the future reference
and to determine accountability. When the customer is in need of finding the ``go to'' person for
a specific function of the project, this will be the person to consult. Of course, this
is not to say that each team member works in a vacuum. As each person's particular skills vary,
one may contract out some part of his own function to another team member. In fact, as many
functions will likely interface with one another, it will almost certainly be required for some
cross-functional cooperation. However, it is important for the person with ownership to be very
particular about the design and understanding of his function.

Each person is responsible for unit testing his own projects.

\section{Portfolio Functionality}
Nick Palumbo will be taking responsibility for the portfolio functionality of this project. One main
aspect of this is the way a user is able to enact trades within their own portfolio, including basic
transactions such as buy and sell as well as more advanced actions like short/cover and stop/limit.
In addition, Nick will be responsible for the structure of leagues and the way in which users create
and join leagues.

\section{Social Functionality}

Dario Rethage will be taking responsibility for the social functionality of this project. As this
application will involve grouping users into leagues and competition, it is important for users
to be able to interact with one another in the form of viewing one another's portfolios,
leaderboards, and the live activity stream. Dario will be in charge of implementing these functions
in a way that is intuitive and clean as well implementing a commenting/messaging system
between users.

\section{Interactive and Graphical Functionality}

Val Red will be taking responsibility for the interactive and graphical functionality of this project.
This includes being able to view one's own portfolio as well as the finances of his league mates in
an easily maneuverable and manipulatable way. In addition, Val will be in charge of the way
users follow the trends of a company's stock performance and view any other relevant data
pertaining to a company, industry, or user.

\section{Account Functionality}
Eric Cuiffo will be taking responsibility for the account functionality of this project. He will be
heading functions such as registration, Facebook integration, and user profile settings. Also,
he will implement account management operations such as changing passwords or account deletion.
In addition, he will create ways in which users can receive information about their accounts, such
as downloading personal portfolio reports or the e-mail updates described above.

\section{League Functionality}
Jeff Rabinowitz will be taking responsibility for the league functionality of this project. Pertinent
functions here will be the powers and actions available to the league manager of a given league,
such as league settings and player management. Jeff will implement the different league
settings available to a manager, as well the ability to create league-wide announcements and
moderating comments within the league.

\section{Administrative Functionality}

Jeff Adler will be taking responsibility for the administrative functionality of this project. This essentially
covers all the power available to a site administrator. He will be responsible for the ways in which an
administrator can manage leagues, users, and user interactions. He will also generate the way in which
site administrators can view statistics about the website, such as user count, active leagues, transactions,
new accounts, popular stocks, and other relevant data that will dictate product decisions down the road.

\end{document}

