\chapter{Customer Statement of Requirements}

\section{Problem Statement}

Perhaps nothing portrays capitalism better than the Stock Market. The ability for individuals and collectives to gain equity in international corporations, trade that equity, and perhaps even gain a profit, has piqued the imagination of a nation for well over a century. One could even say that owning stock is part and parcel of The American Dream.  

However, there is a barrier that separates this dream from reality for many would-be investors: an understanding of the market. The stock market has myriad intricate ways of bundling and exchanging instruments, most of which will be beyond the ken of an economic novice. An economist may be interested in the differences between Mutual Funds and Exchange-Traded Funds; a banker may have the judgment to decide between a Stop Order and a Market Order. These financial techniques offer greater flexibility and control over investments to experienced investors and scientists, who are masters of the field. The beginner does not care to be bothered by these techniques, as they can turn a straightforward process into an overwhelming headache.

With Capital Games we are interested in developing a learning platform for these students - a stock market simulation program. 

“Capital Games” is marketed at two primary classes of user; students and novice investors, each of whom have different needs. Students require a social aspect to their experience - shared simulation instances with global rules and social features. Novice investors require performance metrics and research tools. Both require interactive tutorials, visualization tools, and email updates, in addition to the core requirement of being able to execute various types of trades.

At its simplest, “Capital Games” is about exchanging stocks and managing investments. This is done through the respective menus for each Research, Trading, and Managing Portfolios. Research allows investors to analyze relevant financial metrics of publically traded corporations. Trading allows investors to place market, stop, and limit orders for their various portfolios. Managing Portfolios allows investors to view their investments and performance metrics for each of their simulations. In all menus, data can be visualized and interactively examined, in addition to being tabulated. This unprecedented level of accessibility will ease accessibility to market trend analysis.

Portfolios and trades only exist in the context of “leagues”, or market simulation instances. Each league has with its own rules, administrators, and varying privacy levels. Investors can participate in both public leagues, which anyone can join but offer less social interactivity, and private leagues, which require private email or Facebook invitations but which have expanded social features. Leagues are social because they include Trade Streams of executed trades from league members, Investor Profiles containing trade history and portfolio performance of investors, and a Comments Board. Additionally, each league will have a scoreboard for its members’ portfolio performances. Top investors will have their names and net worth displayed prominently on league pages. 

As a site with social content, it is also important to have the ability to moderate and review submissions by users. This is provided by having two classes of moderators, League Managers and Site Administrators. League Managers are, by default, the users who create a given league, and can ban, invite, and promote users within their leagues, as well as being able to delete comments and create league-wide announcements. League Managers, by default, are also participating in a given league. Site Administrators can delete leagues, ban users and delete comments, add front page announcements, view reports about abusive users, and view other various statistics about users, trades, and leagues. As mentioned previously, even tighter social network integration is a long-term goal.

With the continuing influx of mobile browsing and computing devices to the personal computing market, it is increasingly important to have a single unified interface for users. This is accomplished through the use of Responsive Design, in which a single web page automatically and intelligently reflows itself to accomodate devices of any screen size. This revolutionizes the trading experience -- users don't want to use a dozen odd applications and browsers to access their favorite sites, they want to just click-and-go. These changes are made possible by improvements in mobile browsers, which now universally support Javascript. Therefore one site really will be enough for all users.

As alluded to previously, a strong emphasis of this platform is the use of interactive portfolio graphs. Previous systems have failed to speak directly to users because they presented static images that were impossible to manipulate or interact with. This is a core design feature of Capital Games. We employ the newest, most state-of-the-art graphing tools to allow a user to see any and every stock and portfolio over an indefinite time span with the finest degree of granularity.

Another way of enhancing user experience is by letting users opt to receive daily or weekly email updates about their portfolio performance. This is a feature set that all financial investors have access to and therefore is something that novice investors who use our platform should not be denied. Previous learning platforms have failed to develop a respectable e-mail service -- their demos barely covered assets, and certainly didn't mention trends. We will provide the first fully functional e-mail system.

These features, together with other core capabilities such as email updates and interactive tutorials, provide the most cutting-edge and modern platform for both individual and collaborative efforts to conduct financial simulations.
