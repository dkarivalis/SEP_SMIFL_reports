\section{Economic and Mathematical Models}
\label{econmodels}
\subsection{Perfect Competition}

One of the prevalent concepts in the stock market is the economic concept of perfect
competition, which says that not any single participant has enough resources/power
to control the market.  To apply the concept of perfect competition to our project
we will need the following requirements:

\begin{itemize}
\item
Not one person can control the market or industries, segment, etc.
\item
Users can feel free to execute trades at their convenience without having to worry
about extra costs
\item
Every individual has access to same stock information as other investors
\item
The selling price is the same as the buying price.
\end{itemize}


In the real world, none of these requirements can be met, as there is always some
problem that prevents the market from being in perfect competition.  The following
are just some of the problems:

\begin{itemize}
\item
There are high net worth individuals/companies who have enough capital to change the
tide of a certain sector of the market.  If one of these individuals suddenly decides
to leave a particular market, the move may suddenly shift the market and effect other
investors in that market.
\item
In the real world, users typically don’t have direct access to stocks. They have a
broker (electronic or human) who they interact with, who then have direct access
to stocks.  Users can’t usually execute trades/buy stocks without worrying
about extra costs because of the commissions charged by brokers when trading
stocks.
\item
The world is not a fair place, and neither is the stock market. There are individuals
who because of the field that they work in, have much more insight into a particular
industry/stock.  These individuals then sell this information to potential buyers in
hopes that it gives them an edge in trading. This gives a huge disadvantage to those
that don’t have access to more information bout stocks.
\item
Lastly, in the real world, the selling price is never usually the same as the bid
price.  The Bid-Ask spread, the difference between the buying and selling price
tends to be greater than 0.
\end{itemize}

All these factors lead the stock market away from perfect competition.\\

How do we plan to fix these issues to ensure a near-perfect competition?

\begin{itemize}
\item
All investors start with the same amount of money, this way no one person by default
has more power than anyone else
\item
No commission will be charged when the trades are executed for any investor
\item
Insider trading will be avoided by standardizing the stock information across the board
\item
The ask-bid spread will be 0, so the selling price is the same as the buying price
\end{itemize}

Mathematical Model:

\begin{itemize}
\item
Stock Prices
\begin{itemize}
\item
There are no complicated mathematical models behind how the stock prices are
determined in our platform.  The market prices that are retrieved from Yahoo Finance
are the prices that are available to users in Paramount Investments
\end{itemize}
\item
Achievements
\begin{itemize}
\item
Achievements in Paramount Investments each have their own mathematical model.
There are no complicated algorithms behind how these achievements are attained.
If the user has met the required conditions for a certain achievement, then they
will be given that specific award.
\item
For example: Buy stocks whose P/E Ratio > 1
\end{itemize}
\end{itemize}

