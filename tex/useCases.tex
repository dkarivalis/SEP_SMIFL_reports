% This file contains only the fully dressed use cases
\label{useCases}

\subsection{Preface}

Users will have instant access to the functionality of the site as soon as they
have created an account and logged in. That is, they can perform all of the functions
that an investor can perform. They will also have the ability to choose to create or
join a league, though they are not required to in order to experience the full
functionality of the software. This will give our software a broader demographic
when compared to that of years prior. That being said, league creation and
administrative user delegation will be an important part of the functionality of
the program and will be described its own use case. Core functionalities of respect
user types will be elaborated below.\\

\subsection{Fully-Dressed Use Cases}

Having an account within our database is necessary for the user to experience
functionality within the software. That being said, the user can create an account in
different ways. They can choose to have the information imported by logging in through
any OpenID/OAuth service service (eg: google, facebook, twitter, etc.) This will populate
the database fields automatically with data driven from the external resources.
Once the user has created an account, they can log in through OpenID/OAuth and will
generally remain logged into the system as long as they remain logged into their service
of their choice.\\

\begin{centering}
\renewcommand\arraystretch{1.3} % Causes rows to be spaced
\label{UC-1}
\begin{longtable}{|p{1.2in} p{5in}|}

\hline
\bfseries{\color{color1}Use Case UC-1} &
\bfseries{\color{color1}Register/Create an account using OpenID/OAuth2} \\
\hline
Related Requirements: & ST-1, ST-2 \\
Initiating Actor:     & Guest \\
Actor's Goal:         & Register with our servers\\
Participating Actors: & Guest, Database\\
Preconditions:        & -Guest must not be a registered user\\
Postconditions:       & -The \textbf{Database} is updated with guests information and
                         logs the guess in as an \textbf{Investor}\\
\hline
\multicolumn{2}{|c|}{\color{color1}Flow of Events for Main Success Scenario:}\\
\hline
$\rightarrow$ & 1. \textbf{Guest} navigates to Paramount Investment League and logs in \\
 $\leftarrow$ & 2. \textbf{System} checks \textbf{database} for \textbf{investor} and
isn't found\\
 $\leftarrow$ & 3. \textbf{System} retrieves OpenID/OAuth info and registers guest
in \textbf{database} as an \textbf{investor}\\
$\rightarrow$ & 4. \textbf{System} sends out confimation to user and displays starter
portfolio \\
\hline
\multicolumn{2}{|c|}{\color{color1}Flow of Events for Alternate Scenarios:} \\
\hline
$\rightarrow$ & 1.  \textbf{Investor} attempts to login  \\
 $\leftarrow$ & 2. \textbf{System} checks \textbf{Database} and finds \textbf{investor}\\
 $\leftarrow$ & 3. \textbf{System} loads \textbf{investor} info from \textbf{database} \\
$\rightarrow$ & 4. \textbf{System} displays users portfolio \\
\hline
\end{longtable}
\end{centering}

Any user has the option at any time to create or join a league. The user who has requested
to create a league will have elevated privileges versus a standard user. The league manager
will be prompted to make a league with various setting options for victory condition, badges,
achievements, etc. The league manager can also choose whether or not to make the league private.
A private league will restrict users to those invited by the league manager, and will require
a password to join. After initial setup, league managers will have minimal access to settings.
That is, halfway through a league, the manager cannot decide to change the victory conditions.
This prevents the league manager from abusing power to tip the scales in his or her favor.\\

\begin{centering}
\renewcommand\arraystretch{1.3}
\label{UC-2}
\begin{longtable}{|p{1.2in} p{5in}|}
\hline
\bfseries{\color{color1}Use Case UC-2} &
\bfseries{\color{color1}Create/Join a League} \\
\hline
Related Requirements: & ST-3, ST-8, ST-16, ST-17, ST-18\\
Initiating Actor:     & Investor \\
Actor's Goal:         & Create or join a league to compete in\\
Participating Actors: & Database, other Investors \\
Preconditions:        & -Investor is logged in \\
                      & -league is not created or user hasn't joined league \\
Postconditions:       & -The league is created with the appropriate settings or \\
                      & -The \textbf{Investor} has joined the league \\
                      & -The \textbf{Database} has been updated \\
\hline
\multicolumn{2}{|c|}{\color{color1}Flow of Events for Main Success Scenario:}\\
\hline

$\rightarrow$ & 1. \textbf{Investor} navigates to and clicks on the create league dialogue. \\
$\rightarrow$ & 2. \textbf{System} displays to the \textbf{Investor} the available options for
creating a league.  \\
$\rightarrow$ & 3. \textbf{Investor} updates the settings, such as privacy, league name, number of spots, and managing users \\
$\leftarrow$ & 4. \textbf{System} sends the updated settings to the \textbf{Database} \\
$\rightarrow$ & 5. \textbf{System} sends confimation to the \textbf{Investor} \\
\hline

\multicolumn{2}{|c|}{\color{color1}Flow of Events for Alternate Scenarios:} \\
\hline

\multicolumn{2}{|p{6.2in}|}{3a. The \textbf{Investor} selects league settings that are disallowed,
such as a league name that already exists.} \\ \hline

$\rightarrow$ & 4. \textbf{System} informs user what settings are incorrect.\\ \hline

\multicolumn{2}{|p{6.2in}|}{\textbf{Investor} wishes to join a league} \\
\hline

$\rightarrow$ & 1. \textbf{Investor} navigates to league listing.\\
$\leftarrow$ & 2. \textbf{System} updates \textbf{database} with \textbf{investors} info.\\
$\rightarrow$ & 3. \textbf{System} confirms \textbf{investor} as part of league and displays
league site.\\
\hline
\end{longtable}
\end{centering}

Users can view raw market data which will be pulled from Yahoo Finance in near real time.
Users can view company data either on their own portfolio page or through the company’s
specific info page. That is, the user can view detailed information of each stock or company
before committing to a trade from a variety of sources. Users will be able to compare their
portfolio’s performance to typical market trends from the Nasdaq, S\&P 500, and DJIA. There
will also be a stock ticker ribbon on the bottom of the screen for users to receive constant
real time feeds of most recent trades happening within the market.\\ \\

\begin{centering}
\renewcommand\arraystretch{1.3}
\label{UC-3}
\begin{longtable}{|p{1.2in} p{5in}|}
\hline
\bfseries{\color{color1}Use Case UC-3} &
\bfseries{\color{color1}View Market Data} \\
\hline
Related Requirements: & ST-4, ST-5, ST-10, ST-11 \\
Initiating Actor:     & Investor \\
Actor's Goal:         & View the latest information about stocks, companies, and trades\\
Participating Actors: & Database, Yahoo! Finance \\
Preconditions:        & -Yahoo! Finance is accepting inquiries \\
                      & -Investor is logged in \\
Postconditions:       & -None worth mentioning \\

\hline
\multicolumn{2}{|c|}{\color{color1}Flow of Events for Main Success Scenario:}\\
\hline

$\rightarrow$ & 1. \textbf{Investor} searches for a market term  \\
 $\leftarrow$ & 2. \textbf{System} sends request to \textbf{database} \\
$\rightarrow$ & 3. \textbf{System} returns suggested terms \\
$\rightarrow$ & 4. \textbf{Investor} selects a term from suggested terms list and sends request \\
$\leftarrow$ & 5. \textbf{System} sends request to \textbf{Yahoo! Finance} \\
$\leftarrow$ & 6. \textbf{database} is updated. \\
\hline

\multicolumn{2}{|c|}{\color{color1}Flow of Events for Alternate Scenarios:} \\
\hline

\multicolumn{2}{|p{6.2in}|}{Search Fails} \\
\hline

$\leftarrow$ & 6. \textbf{Yahoo! Finance} returns no results \\
$\rightarrow$ & 7. \textbf{System} informs \textbf{investor} of search failure \\

\hline

\end{longtable}
\end{centering}

User will be able to view all major items within their portfolio as well as place trades from
their portfolio page. From this page, a user can view detailed analysis and graphs of each
of their respective stocks as well as their current rank within their league (if applicable)
and globally. Users will also be able to place trades for respective companies through their
portfolio page. Users can buy, sell, short, or carry out any additional action on any stock
or security within the limits of their finances and league settings through this page (See UC-5).
Users will also be able to customize and change views as well as add stock index comparisons
to monitor their success vs market success.\\

\begin{centering}
\label{UC-4}
\renewcommand\arraystretch{1.3}
\begin{longtable}{|p{1.2in} p{5in}|}
\hline
\bfseries{\color{color1}Use Case UC-4} &
\bfseries{\color{color1}Manage Portfolio} \\
\hline
Related Requirements: & ST-8, ST-9, ST-10, ST-12, ST-13, ST-14 \\
Initiating Actor:     & Investor \\
Actor's Goal:         & Manage portfolio by viewing current standings/stocks/securities \\
Participating Actors: & Database, Yahoo! Finance \\
Preconditions:        & -Yahoo! Finance is accepting inquiries \\
                      & -User is logged in \\
Postconditions:       & -\textbf{Investor}'s portfolio is updated to reflect change
                        in position \\
\hline
\multicolumn{2}{|c|}{\color{color1}Flow of Events for Main Success Scenario:}\\
\hline
$\rightarrow$ & 1. \textbf{User} selects the league in which they would like to place the order \\
$\leftarrow$ & 2. \textbf{System} displays prompt for market order, including type, amount, and company \\
$\rightarrow$ & 3. \textbf{User} fills out form and requests the order be placed \\
$\leftarrow$ & 4. \textbf{System} (a) requests market price from \textbf{Yahoo! Finance} and (b) places the order into the \textbf{Database} \\
$\leftarrow$ & 5. The order either resolves or expires, and the \textbf{System} updates the \textbf{User}'s position in the \textbf{Database} accordingly \\
\hline
\multicolumn{2}{|c|}{\color{color1}Flow of Events for Extensions (Alternate Scenarios):} \\
\hline

\multicolumn{2}{|p{6.2in}|}{1a. The \textbf{User} chooses to place a market order from a company's profile rather than from the league page} \\
\hline
$\rightarrow$ & 1. The \textbf{User} selects which league in which to place the order  \\
$\leftarrow$ & 2. The \textbf{System} takes the \textbf{User} to league marker order prompt as described in Step 2 above, with the prompt for company already filled out \\
$\rightarrow$ & 3. Go to Step 3 above \\
\hline
\multicolumn{2}{|p{6.2in}|}{4a. The \textbf{User} does not have enough money or margin to place the order} \\
\hline
$\leftarrow$ & 1. \textbf{System} informs the \textbf{User} that they do not have enough money or margin to place the order and returns them to the market order prompt \\
\hline 
\end{longtable}
\end{centering}

User should be able to place trades from various locations. That is, they may place it through
their portfolio by typing in the ticker and quantity of shares. They may also navigate to a
certain company’s page and elect to purchase shares there. Selling shares should be done
through the user’s portfolio where they may see the exact quantity of shares of each respective
companies they own. Error messages will be thrown and orders not processed should a user request
to buy more shares of a company the he or she can afford or the user attempts to sell more than
he or she has. Main transactions will occur through the user’s portfolio. \\ \\

\begin{centering}
\label{UC-5}
\renewcommand\arraystretch{1.3}
\begin{longtable}{|p{1.2in} p{5in}|}
\hline
\bfseries{\color{color1}Use Case UC-5} &
\bfseries{\color{color1}Place a Market Order} \\
\hline
Related Requirements: & ST-6, ST-11 \\
Initiating Actor:     & Investor \\
Actor's Goal:         & Place orders to buy/sell/short stocks, or place a stop/limit order \\
Participating Actors: & Database, Yahoo! Finance API \\
Preconditions:        & -Investor is logged in \\
                      & -Yahoo! Finance is accepting inquiries \\
Postconditions:       & -\textbf{Database} us updated with the users position \\
\hline
\multicolumn{2}{|c|}{\color{color1}Flow of Events for Main Success Scenario:}\\
\hline
$\rightarrow$ & 1. \textbf{User} selects a league member's profile \\
$\leftarrow$ & 2. \textbf{System} requests that member's information from the \textbf{Database} and displays it in an organized and graphical manner to the \textbf{User} \\
\hline
\multicolumn{2}{|c|}{\color{color1}Flow of Events for Extensions (Alternate Scenarios):} \\
\hline
\multicolumn{2}{|p{6.2in}|}{2a. \textbf{User} is viewing their own portfolio} \\
\hline
$\leftarrow$ & 1. \textbf{System} gives the \textbf{User} options to place market orders related to their existing positions \\
\hline
\end{longtable}
\end{centering}

Of the 3 user types, administrator is the highest and reserved only for developers and
administrators of the software. Administrators have the ability to modify or delete leagues
or specific users if the administrator feels that power is being abused. The administrator
will also have elevated privileges to makes changes to the site. Their main purpose will be to
suspend or ban users or leagues and ensure that the site is not being abused. This includes but
is not limited to robot or AI users or user account spamming or advertising rather than trading
properly. \\

\begin{centering}
\renewcommand\arraystretch{1.3} % Causes rows to be spaced
\label{UC-6}
\begin{longtable}{|p{1.2in} p{5in}|}
\hline
\bfseries{\color{color1}Use Case UC-6} &
\bfseries{\color{color1}Take Administrative Actions} \\
\hline
Related Requirements: & ST-19, ST-20, ST-21 \\
Initiating Actor:     & Site Administrator \\
Actor's Goal:         & Perform administrative work for the website, manage database \\
Participating Actors: & Database, Investors, League Manager \\
Preconditions:        & -User is the site Administrator \\
                      & -Administrative actions need to be taken \\
Postconditions:       & -Conflicts/Issues have been resolved \\
\hline
\multicolumn{2}{|c|}{\color{color1}Flow of Events for Main Success Scenario:}\\
\hline
$\rightarrow$ & 1. \textbf{User} selects the tutorial option from the site's main page \\
$\leftarrow$ & 2. \textbf{System} displays possible topics on which the \textbf{User} may be educated on \\
$\rightarrow$ & 3. \textbf{User} selects topic \\
$\leftarrow$ & 4. \textbf{System} presents an interactive tutorial to the \textbf{User}, which will be further elaborated upon in a later section \\
\hline
\end{longtable}
\end{centering}

In order to maintain a clean fantasy finance experience for our regular users, site
administrators will reserve the ability to moderate other users--issuing warnings,
suspensions, or even bans for abusive activity. To put it explicitly: \\

\textbf{CG-BP04:} Site administrators will warn, suspend, or ban users for abusive
activity--this includes aggressive behavior on league comments or user messages,
spamming users, joining numerous leagues without active participation, and anything
else that is deemed to harm the experience for other users. \\


\begin{centering}
\renewcommand\arraystretch{1.3} % Causes rows to be spaced
\label{UC-7}
\begin{longtable}{|p{1.2in} p{5in}|}
\hline
\bfseries{\color{color1}Use Case UC-7} &
\bfseries{\color{color1}Manage League Settings} \\
\hline
Related Requirements: & ST-16, ST-17, ST-18 \\
Initiating Actor:     & League Manager \\
Actor's Goal:         & Change league settings to the League Mangers preference \\
Participating Actors: & Database, other Investors \\
Preconditions:        & -Initiating actor is the \textbf{League Manager} \\
                      & -There are outstanding abuse reports \\
Postconditions:       & -The \textbf{Database} is updated to reflect the chagnes made. \\
 & The abuse report shows that it has been resolved on the administration page\\
\hline
\multicolumn{2}{|c|}{\color{color1}Flow of Events for Main Success Scenario:}\\
\hline
$\rightarrow$ & 1. \textbf{Site Administrator} selects the site administration page option from the main screen (only viewable by \textbf{Site Administrators}) \\
$\leftarrow$ & 2. \textbf{System} makes a request to the \textbf{Database} and displays all outstanding abuse reports\\
$\rightarrow$ & 3. \textbf{Site Administrator} (a) selects an abuse report, (b) reviews the report, and (c) selects what action is to be taken (if any)\\
$\leftarrow$ & 4. \textbf{System} implements the action selected by the \textbf{Site Administrator} and updates the \textbf{Database} accordingly \\
$\leftarrow$ & 5. \textbf{System} notifies the offending \textbf{User} of any actions taken against them \\
\hline
\end{longtable}
\end{centering}

\subsection{Traceability Matrix}

The traceability matrix presented in \em Figure 3.2 \em is based on only the full
dressed use cases above and thus is only a partial representation of the complete project.\\

\begin{figure}
\centering
\includegraphics[width=5.5in]{./img/traceability.png}
\caption{The traceability matrix for the given use cases.}
\end{figure}

