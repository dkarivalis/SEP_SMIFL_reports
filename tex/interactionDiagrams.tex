\chapter{System Interaction Diagrams}

% This chapter requires a good 1-2 pages
% prefacing the way systems will behave
% and introducing the types of diagrams
% that will be shown.

\section{Introduction}



\section{Diagrams}

\subsection{Use Case 1}

Shown in the sequence diagram for UC-1 begins with two options for the Guest.
Either login or register an account. If a user attempts to register a new
account the Account Controller is contacted with the user’s information. Then
the Account Controller can attempt to check to make sure no duplicate login
information exists in the database via the DB Connection module and if not it
will store the new user information into the database. After this happens the
user will be sent a confirmation email. Then the Account Controller will update
the Login View.\\

If a user attempts to login, the Account Controller will attempt to authenticate
the login details with details found in the database via the DB Connection
module. If the details match correctly then the Account Controller will send the
guest into investor mode and therefore displaying them the Player Profile View.

\begin{figure}[H]
\centering
\includegraphics[width=5.5in]{./img/inter/uc1.jpg}
\caption{UC-1}
\end{figure}

\subsection{Use Case 2}
Shown in the sequence diagram for UC-2 is the flow of how to create an
investment league. When an investor selects to create a league the League
Controller will be contacted. This will update the Player Profile View display
the available options for creating a league. After, there is a function
updateSettings() which will create the league and process it in the database
via the DB Connection and also allow settings to be updated for a league. Not
shown in the diagram is the alternative case of joining a league. The process
to join a league is straightforward, where the league controller will show
available leagues and then if an investor chooses to join they will be entered
into the list in the database to associate with this league.

\begin{figure}[H]
\centering
\includegraphics[width=5.5in]{./img/inter/uc2.jpg}
\caption{UC-2}
\end{figure}

\subsection{Use Case 3}
Viewing market data is accomplished by an investor searching a term. The Order
System Controller then finds this term which is most likely a company name or
stock symbol. The system will fetch matches from the database via the DB
Connection module and display them from the user. The investor will choose a
match. The Order System Controller takes the chosen term and requests its data
from the Yahoo! Finance API via the Yahoo! Finance Adapter. The Order System
Controller will update the database via the DB Connection module for this term,
and then continue to show the Market Data View.\\


\begin{figure}[H]
\centering
\includegraphics[width=5.5in]{./img/inter/uc3.jpg}
\caption{UC-3}
\end{figure}

\subsection{Use Case 4}
The investor should be able to view and make changes to their Portfolio View.
When the user clicks to show portfolio, the Portfolio Controller will fetch the
investor’s portfolio stocks from the database via the DB Connection module. The
investor can also update their view of the portfolio and other settings.

\begin{figure}[H]
\centering
\includegraphics[width=5.5in]{./img/inter/uc4.jpg}
\caption{UC-4}
\end{figure}

\subsection{Use Case 5}
The investor needs to be able to place market orders. As soon as the investor
places an order the Order System Controller contacts Yahoo! Finance API via the
Yahoo! Finance Adapter to retrieve the current price of the stock. After the
current price is found the Order System Controller must confirm with the
database via the DB Connection module that the user has enough funds to make a
buy order or enough stock to make the sell order. After the trade is confirmed
information will be stored about it in the database via the DB Connection
module and the changes will be displayed in the investor’s portfolio.

\begin{figure}[H]
\centering
\includegraphics[width=5.5in]{./img/inter/uc5.jpg}
\caption{UC-5}
\end{figure}


\section{Alternate Solution Diagramming}

Software design shouldn't be about picking your first idea and going with it.
You need to consider alternative solutions to the task at hand and pick the best
one based on the known criteria.  For this reason we are documenting some of
our alternative solutions for historical reference.\\

\begin{figure}[H]
\centering
\includegraphics[width=5.5in]{./img/inter/alt1.png}
\caption{UC-1 alternate solution considered}
\end{figure}
\begin{figure}[H]
\centering
\includegraphics[width=5.5in]{./img/inter/alt2.png}
\caption{UC-2 alternate solution considered}
\end{figure}
\begin{figure}[H]
\centering
\includegraphics[width=5.5in]{./img/inter/alt3.png}
\caption{UC-3 alternate solution considered}
\end{figure}
\begin{figure}[H]
\centering
\includegraphics[width=5.5in]{./img/inter/alt4.png}
\caption{UC-4 alternate solution considered}
\end{figure}
\begin{figure}[H]
\centering
\includegraphics[width=5.5in]{./img/inter/alt5.png}
\caption{UC-5 alternate solution considered}
\end{figure}
