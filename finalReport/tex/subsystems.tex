\section{Identifying Subsystems}

Paramount investments aims to set its platform on multiple interfaces.
As such, subsystem identification becomes an integral part of initial
analysis of the software. On a thick layer, our platform exists with
a front-end system and a back-end system. But on a much deeper level,
we can see that, each of these subsystems can be broken down into
still greater detail. Front end systems typically involve user
interface, and object interactions with the user. Back-end will refer
to all database schema, implementation and interactions with relevant
hardware. Also included are non-associative items which are necessary
to the success of our system.\\

Front-end systems are formally plain. The user interface which displays
views and specific data to the user on multiple platform is included
here. That is, it will contain different mappings and specific
implementations for iOS and Android as well as natively for the Web.
The front-end system will have to maintain constant communication
with the back-end system to maintain consistency and retrieve data
regularly. It needs to be able to successfully communicate information
from commands given by the user and communicate them to the back end.
The back end system will retrieve necessary data and information and
return the data to the UI and user to project the page or information
requested.\\

Our back end system will be broken down further and is easily
considered the most important part of our infrastructure. Since
we are using the MVC framework, the back end system is to be broken
down into controller and database subsystems. Additionally, we will
have the financial retrieval system and queuing systems as previously
outline. Thus, the bulk of the command processing is handled by our
back-end subsystem.  The back-end system must not only communicate
among the subsystems within itself, but it must also communicate with
the front-end UI system to respond to commands and also communicate
with the non-associate systems as well.\\

Breaking down the subsystem further, we highlight the importance
of the financial retrieval system, and the queue system. The
financial retrieval system will communicate with Yahoo! Finance
to retrieve relevant information as requested by the controller
(whenever the controller receives an input from the front-end user).
The queuing system will handle other processes and largely
communication with non-associative systems. It will also be
involved in queuing and handling all back-end processes and
monitors to ensure that the correct commands are processed at the
correct time. The success of these modules, the success of the entire
back-end system, and the success of communication amongst the
systems will be crucial for the overall success of the software.

